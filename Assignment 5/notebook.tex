
% Default to the notebook output style

    


% Inherit from the specified cell style.




    
\documentclass[11pt]{article}

    
    
    \usepackage[T1]{fontenc}
    % Nicer default font (+ math font) than Computer Modern for most use cases
    \usepackage{mathpazo}

    % Basic figure setup, for now with no caption control since it's done
    % automatically by Pandoc (which extracts ![](path) syntax from Markdown).
    \usepackage{graphicx}
    % We will generate all images so they have a width \maxwidth. This means
    % that they will get their normal width if they fit onto the page, but
    % are scaled down if they would overflow the margins.
    \makeatletter
    \def\maxwidth{\ifdim\Gin@nat@width>\linewidth\linewidth
    \else\Gin@nat@width\fi}
    \makeatother
    \let\Oldincludegraphics\includegraphics
    % Set max figure width to be 80% of text width, for now hardcoded.
    \renewcommand{\includegraphics}[1]{\Oldincludegraphics[width=.8\maxwidth]{#1}}
    % Ensure that by default, figures have no caption (until we provide a
    % proper Figure object with a Caption API and a way to capture that
    % in the conversion process - todo).
    \usepackage{caption}
    \DeclareCaptionLabelFormat{nolabel}{}
    \captionsetup{labelformat=nolabel}

    \usepackage{adjustbox} % Used to constrain images to a maximum size 
    \usepackage{xcolor} % Allow colors to be defined
    \usepackage{enumerate} % Needed for markdown enumerations to work
    \usepackage{geometry} % Used to adjust the document margins
    \usepackage{amsmath} % Equations
    \usepackage{amssymb} % Equations
    \usepackage{textcomp} % defines textquotesingle
    % Hack from http://tex.stackexchange.com/a/47451/13684:
    \AtBeginDocument{%
        \def\PYZsq{\textquotesingle}% Upright quotes in Pygmentized code
    }
    \usepackage{upquote} % Upright quotes for verbatim code
    \usepackage{eurosym} % defines \euro
    \usepackage[mathletters]{ucs} % Extended unicode (utf-8) support
    \usepackage[utf8x]{inputenc} % Allow utf-8 characters in the tex document
    \usepackage{fancyvrb} % verbatim replacement that allows latex
    \usepackage{grffile} % extends the file name processing of package graphics 
                         % to support a larger range 
    % The hyperref package gives us a pdf with properly built
    % internal navigation ('pdf bookmarks' for the table of contents,
    % internal cross-reference links, web links for URLs, etc.)
    \usepackage{hyperref}
    \usepackage{longtable} % longtable support required by pandoc >1.10
    \usepackage{booktabs}  % table support for pandoc > 1.12.2
    \usepackage[inline]{enumitem} % IRkernel/repr support (it uses the enumerate* environment)
    \usepackage[normalem]{ulem} % ulem is needed to support strikethroughs (\sout)
                                % normalem makes italics be italics, not underlines
    

    
    
    % Colors for the hyperref package
    \definecolor{urlcolor}{rgb}{0,.145,.698}
    \definecolor{linkcolor}{rgb}{.71,0.21,0.01}
    \definecolor{citecolor}{rgb}{.12,.54,.11}

    % ANSI colors
    \definecolor{ansi-black}{HTML}{3E424D}
    \definecolor{ansi-black-intense}{HTML}{282C36}
    \definecolor{ansi-red}{HTML}{E75C58}
    \definecolor{ansi-red-intense}{HTML}{B22B31}
    \definecolor{ansi-green}{HTML}{00A250}
    \definecolor{ansi-green-intense}{HTML}{007427}
    \definecolor{ansi-yellow}{HTML}{DDB62B}
    \definecolor{ansi-yellow-intense}{HTML}{B27D12}
    \definecolor{ansi-blue}{HTML}{208FFB}
    \definecolor{ansi-blue-intense}{HTML}{0065CA}
    \definecolor{ansi-magenta}{HTML}{D160C4}
    \definecolor{ansi-magenta-intense}{HTML}{A03196}
    \definecolor{ansi-cyan}{HTML}{60C6C8}
    \definecolor{ansi-cyan-intense}{HTML}{258F8F}
    \definecolor{ansi-white}{HTML}{C5C1B4}
    \definecolor{ansi-white-intense}{HTML}{A1A6B2}

    % commands and environments needed by pandoc snippets
    % extracted from the output of `pandoc -s`
    \providecommand{\tightlist}{%
      \setlength{\itemsep}{0pt}\setlength{\parskip}{0pt}}
    \DefineVerbatimEnvironment{Highlighting}{Verbatim}{commandchars=\\\{\}}
    % Add ',fontsize=\small' for more characters per line
    \newenvironment{Shaded}{}{}
    \newcommand{\KeywordTok}[1]{\textcolor[rgb]{0.00,0.44,0.13}{\textbf{{#1}}}}
    \newcommand{\DataTypeTok}[1]{\textcolor[rgb]{0.56,0.13,0.00}{{#1}}}
    \newcommand{\DecValTok}[1]{\textcolor[rgb]{0.25,0.63,0.44}{{#1}}}
    \newcommand{\BaseNTok}[1]{\textcolor[rgb]{0.25,0.63,0.44}{{#1}}}
    \newcommand{\FloatTok}[1]{\textcolor[rgb]{0.25,0.63,0.44}{{#1}}}
    \newcommand{\CharTok}[1]{\textcolor[rgb]{0.25,0.44,0.63}{{#1}}}
    \newcommand{\StringTok}[1]{\textcolor[rgb]{0.25,0.44,0.63}{{#1}}}
    \newcommand{\CommentTok}[1]{\textcolor[rgb]{0.38,0.63,0.69}{\textit{{#1}}}}
    \newcommand{\OtherTok}[1]{\textcolor[rgb]{0.00,0.44,0.13}{{#1}}}
    \newcommand{\AlertTok}[1]{\textcolor[rgb]{1.00,0.00,0.00}{\textbf{{#1}}}}
    \newcommand{\FunctionTok}[1]{\textcolor[rgb]{0.02,0.16,0.49}{{#1}}}
    \newcommand{\RegionMarkerTok}[1]{{#1}}
    \newcommand{\ErrorTok}[1]{\textcolor[rgb]{1.00,0.00,0.00}{\textbf{{#1}}}}
    \newcommand{\NormalTok}[1]{{#1}}
    
    % Additional commands for more recent versions of Pandoc
    \newcommand{\ConstantTok}[1]{\textcolor[rgb]{0.53,0.00,0.00}{{#1}}}
    \newcommand{\SpecialCharTok}[1]{\textcolor[rgb]{0.25,0.44,0.63}{{#1}}}
    \newcommand{\VerbatimStringTok}[1]{\textcolor[rgb]{0.25,0.44,0.63}{{#1}}}
    \newcommand{\SpecialStringTok}[1]{\textcolor[rgb]{0.73,0.40,0.53}{{#1}}}
    \newcommand{\ImportTok}[1]{{#1}}
    \newcommand{\DocumentationTok}[1]{\textcolor[rgb]{0.73,0.13,0.13}{\textit{{#1}}}}
    \newcommand{\AnnotationTok}[1]{\textcolor[rgb]{0.38,0.63,0.69}{\textbf{\textit{{#1}}}}}
    \newcommand{\CommentVarTok}[1]{\textcolor[rgb]{0.38,0.63,0.69}{\textbf{\textit{{#1}}}}}
    \newcommand{\VariableTok}[1]{\textcolor[rgb]{0.10,0.09,0.49}{{#1}}}
    \newcommand{\ControlFlowTok}[1]{\textcolor[rgb]{0.00,0.44,0.13}{\textbf{{#1}}}}
    \newcommand{\OperatorTok}[1]{\textcolor[rgb]{0.40,0.40,0.40}{{#1}}}
    \newcommand{\BuiltInTok}[1]{{#1}}
    \newcommand{\ExtensionTok}[1]{{#1}}
    \newcommand{\PreprocessorTok}[1]{\textcolor[rgb]{0.74,0.48,0.00}{{#1}}}
    \newcommand{\AttributeTok}[1]{\textcolor[rgb]{0.49,0.56,0.16}{{#1}}}
    \newcommand{\InformationTok}[1]{\textcolor[rgb]{0.38,0.63,0.69}{\textbf{\textit{{#1}}}}}
    \newcommand{\WarningTok}[1]{\textcolor[rgb]{0.38,0.63,0.69}{\textbf{\textit{{#1}}}}}
    
    
    % Define a nice break command that doesn't care if a line doesn't already
    % exist.
    \def\br{\hspace*{\fill} \\* }
    % Math Jax compatability definitions
    \def\gt{>}
    \def\lt{<}
    % Document parameters
    \title{2019ws-BCIIL-Sheet05-Karastoyanov,Mohan,Al-Asadi}
    
    
    

    % Pygments definitions
    
\makeatletter
\def\PY@reset{\let\PY@it=\relax \let\PY@bf=\relax%
    \let\PY@ul=\relax \let\PY@tc=\relax%
    \let\PY@bc=\relax \let\PY@ff=\relax}
\def\PY@tok#1{\csname PY@tok@#1\endcsname}
\def\PY@toks#1+{\ifx\relax#1\empty\else%
    \PY@tok{#1}\expandafter\PY@toks\fi}
\def\PY@do#1{\PY@bc{\PY@tc{\PY@ul{%
    \PY@it{\PY@bf{\PY@ff{#1}}}}}}}
\def\PY#1#2{\PY@reset\PY@toks#1+\relax+\PY@do{#2}}

\expandafter\def\csname PY@tok@w\endcsname{\def\PY@tc##1{\textcolor[rgb]{0.73,0.73,0.73}{##1}}}
\expandafter\def\csname PY@tok@c\endcsname{\let\PY@it=\textit\def\PY@tc##1{\textcolor[rgb]{0.25,0.50,0.50}{##1}}}
\expandafter\def\csname PY@tok@cp\endcsname{\def\PY@tc##1{\textcolor[rgb]{0.74,0.48,0.00}{##1}}}
\expandafter\def\csname PY@tok@k\endcsname{\let\PY@bf=\textbf\def\PY@tc##1{\textcolor[rgb]{0.00,0.50,0.00}{##1}}}
\expandafter\def\csname PY@tok@kp\endcsname{\def\PY@tc##1{\textcolor[rgb]{0.00,0.50,0.00}{##1}}}
\expandafter\def\csname PY@tok@kt\endcsname{\def\PY@tc##1{\textcolor[rgb]{0.69,0.00,0.25}{##1}}}
\expandafter\def\csname PY@tok@o\endcsname{\def\PY@tc##1{\textcolor[rgb]{0.40,0.40,0.40}{##1}}}
\expandafter\def\csname PY@tok@ow\endcsname{\let\PY@bf=\textbf\def\PY@tc##1{\textcolor[rgb]{0.67,0.13,1.00}{##1}}}
\expandafter\def\csname PY@tok@nb\endcsname{\def\PY@tc##1{\textcolor[rgb]{0.00,0.50,0.00}{##1}}}
\expandafter\def\csname PY@tok@nf\endcsname{\def\PY@tc##1{\textcolor[rgb]{0.00,0.00,1.00}{##1}}}
\expandafter\def\csname PY@tok@nc\endcsname{\let\PY@bf=\textbf\def\PY@tc##1{\textcolor[rgb]{0.00,0.00,1.00}{##1}}}
\expandafter\def\csname PY@tok@nn\endcsname{\let\PY@bf=\textbf\def\PY@tc##1{\textcolor[rgb]{0.00,0.00,1.00}{##1}}}
\expandafter\def\csname PY@tok@ne\endcsname{\let\PY@bf=\textbf\def\PY@tc##1{\textcolor[rgb]{0.82,0.25,0.23}{##1}}}
\expandafter\def\csname PY@tok@nv\endcsname{\def\PY@tc##1{\textcolor[rgb]{0.10,0.09,0.49}{##1}}}
\expandafter\def\csname PY@tok@no\endcsname{\def\PY@tc##1{\textcolor[rgb]{0.53,0.00,0.00}{##1}}}
\expandafter\def\csname PY@tok@nl\endcsname{\def\PY@tc##1{\textcolor[rgb]{0.63,0.63,0.00}{##1}}}
\expandafter\def\csname PY@tok@ni\endcsname{\let\PY@bf=\textbf\def\PY@tc##1{\textcolor[rgb]{0.60,0.60,0.60}{##1}}}
\expandafter\def\csname PY@tok@na\endcsname{\def\PY@tc##1{\textcolor[rgb]{0.49,0.56,0.16}{##1}}}
\expandafter\def\csname PY@tok@nt\endcsname{\let\PY@bf=\textbf\def\PY@tc##1{\textcolor[rgb]{0.00,0.50,0.00}{##1}}}
\expandafter\def\csname PY@tok@nd\endcsname{\def\PY@tc##1{\textcolor[rgb]{0.67,0.13,1.00}{##1}}}
\expandafter\def\csname PY@tok@s\endcsname{\def\PY@tc##1{\textcolor[rgb]{0.73,0.13,0.13}{##1}}}
\expandafter\def\csname PY@tok@sd\endcsname{\let\PY@it=\textit\def\PY@tc##1{\textcolor[rgb]{0.73,0.13,0.13}{##1}}}
\expandafter\def\csname PY@tok@si\endcsname{\let\PY@bf=\textbf\def\PY@tc##1{\textcolor[rgb]{0.73,0.40,0.53}{##1}}}
\expandafter\def\csname PY@tok@se\endcsname{\let\PY@bf=\textbf\def\PY@tc##1{\textcolor[rgb]{0.73,0.40,0.13}{##1}}}
\expandafter\def\csname PY@tok@sr\endcsname{\def\PY@tc##1{\textcolor[rgb]{0.73,0.40,0.53}{##1}}}
\expandafter\def\csname PY@tok@ss\endcsname{\def\PY@tc##1{\textcolor[rgb]{0.10,0.09,0.49}{##1}}}
\expandafter\def\csname PY@tok@sx\endcsname{\def\PY@tc##1{\textcolor[rgb]{0.00,0.50,0.00}{##1}}}
\expandafter\def\csname PY@tok@m\endcsname{\def\PY@tc##1{\textcolor[rgb]{0.40,0.40,0.40}{##1}}}
\expandafter\def\csname PY@tok@gh\endcsname{\let\PY@bf=\textbf\def\PY@tc##1{\textcolor[rgb]{0.00,0.00,0.50}{##1}}}
\expandafter\def\csname PY@tok@gu\endcsname{\let\PY@bf=\textbf\def\PY@tc##1{\textcolor[rgb]{0.50,0.00,0.50}{##1}}}
\expandafter\def\csname PY@tok@gd\endcsname{\def\PY@tc##1{\textcolor[rgb]{0.63,0.00,0.00}{##1}}}
\expandafter\def\csname PY@tok@gi\endcsname{\def\PY@tc##1{\textcolor[rgb]{0.00,0.63,0.00}{##1}}}
\expandafter\def\csname PY@tok@gr\endcsname{\def\PY@tc##1{\textcolor[rgb]{1.00,0.00,0.00}{##1}}}
\expandafter\def\csname PY@tok@ge\endcsname{\let\PY@it=\textit}
\expandafter\def\csname PY@tok@gs\endcsname{\let\PY@bf=\textbf}
\expandafter\def\csname PY@tok@gp\endcsname{\let\PY@bf=\textbf\def\PY@tc##1{\textcolor[rgb]{0.00,0.00,0.50}{##1}}}
\expandafter\def\csname PY@tok@go\endcsname{\def\PY@tc##1{\textcolor[rgb]{0.53,0.53,0.53}{##1}}}
\expandafter\def\csname PY@tok@gt\endcsname{\def\PY@tc##1{\textcolor[rgb]{0.00,0.27,0.87}{##1}}}
\expandafter\def\csname PY@tok@err\endcsname{\def\PY@bc##1{\setlength{\fboxsep}{0pt}\fcolorbox[rgb]{1.00,0.00,0.00}{1,1,1}{\strut ##1}}}
\expandafter\def\csname PY@tok@kc\endcsname{\let\PY@bf=\textbf\def\PY@tc##1{\textcolor[rgb]{0.00,0.50,0.00}{##1}}}
\expandafter\def\csname PY@tok@kd\endcsname{\let\PY@bf=\textbf\def\PY@tc##1{\textcolor[rgb]{0.00,0.50,0.00}{##1}}}
\expandafter\def\csname PY@tok@kn\endcsname{\let\PY@bf=\textbf\def\PY@tc##1{\textcolor[rgb]{0.00,0.50,0.00}{##1}}}
\expandafter\def\csname PY@tok@kr\endcsname{\let\PY@bf=\textbf\def\PY@tc##1{\textcolor[rgb]{0.00,0.50,0.00}{##1}}}
\expandafter\def\csname PY@tok@bp\endcsname{\def\PY@tc##1{\textcolor[rgb]{0.00,0.50,0.00}{##1}}}
\expandafter\def\csname PY@tok@fm\endcsname{\def\PY@tc##1{\textcolor[rgb]{0.00,0.00,1.00}{##1}}}
\expandafter\def\csname PY@tok@vc\endcsname{\def\PY@tc##1{\textcolor[rgb]{0.10,0.09,0.49}{##1}}}
\expandafter\def\csname PY@tok@vg\endcsname{\def\PY@tc##1{\textcolor[rgb]{0.10,0.09,0.49}{##1}}}
\expandafter\def\csname PY@tok@vi\endcsname{\def\PY@tc##1{\textcolor[rgb]{0.10,0.09,0.49}{##1}}}
\expandafter\def\csname PY@tok@vm\endcsname{\def\PY@tc##1{\textcolor[rgb]{0.10,0.09,0.49}{##1}}}
\expandafter\def\csname PY@tok@sa\endcsname{\def\PY@tc##1{\textcolor[rgb]{0.73,0.13,0.13}{##1}}}
\expandafter\def\csname PY@tok@sb\endcsname{\def\PY@tc##1{\textcolor[rgb]{0.73,0.13,0.13}{##1}}}
\expandafter\def\csname PY@tok@sc\endcsname{\def\PY@tc##1{\textcolor[rgb]{0.73,0.13,0.13}{##1}}}
\expandafter\def\csname PY@tok@dl\endcsname{\def\PY@tc##1{\textcolor[rgb]{0.73,0.13,0.13}{##1}}}
\expandafter\def\csname PY@tok@s2\endcsname{\def\PY@tc##1{\textcolor[rgb]{0.73,0.13,0.13}{##1}}}
\expandafter\def\csname PY@tok@sh\endcsname{\def\PY@tc##1{\textcolor[rgb]{0.73,0.13,0.13}{##1}}}
\expandafter\def\csname PY@tok@s1\endcsname{\def\PY@tc##1{\textcolor[rgb]{0.73,0.13,0.13}{##1}}}
\expandafter\def\csname PY@tok@mb\endcsname{\def\PY@tc##1{\textcolor[rgb]{0.40,0.40,0.40}{##1}}}
\expandafter\def\csname PY@tok@mf\endcsname{\def\PY@tc##1{\textcolor[rgb]{0.40,0.40,0.40}{##1}}}
\expandafter\def\csname PY@tok@mh\endcsname{\def\PY@tc##1{\textcolor[rgb]{0.40,0.40,0.40}{##1}}}
\expandafter\def\csname PY@tok@mi\endcsname{\def\PY@tc##1{\textcolor[rgb]{0.40,0.40,0.40}{##1}}}
\expandafter\def\csname PY@tok@il\endcsname{\def\PY@tc##1{\textcolor[rgb]{0.40,0.40,0.40}{##1}}}
\expandafter\def\csname PY@tok@mo\endcsname{\def\PY@tc##1{\textcolor[rgb]{0.40,0.40,0.40}{##1}}}
\expandafter\def\csname PY@tok@ch\endcsname{\let\PY@it=\textit\def\PY@tc##1{\textcolor[rgb]{0.25,0.50,0.50}{##1}}}
\expandafter\def\csname PY@tok@cm\endcsname{\let\PY@it=\textit\def\PY@tc##1{\textcolor[rgb]{0.25,0.50,0.50}{##1}}}
\expandafter\def\csname PY@tok@cpf\endcsname{\let\PY@it=\textit\def\PY@tc##1{\textcolor[rgb]{0.25,0.50,0.50}{##1}}}
\expandafter\def\csname PY@tok@c1\endcsname{\let\PY@it=\textit\def\PY@tc##1{\textcolor[rgb]{0.25,0.50,0.50}{##1}}}
\expandafter\def\csname PY@tok@cs\endcsname{\let\PY@it=\textit\def\PY@tc##1{\textcolor[rgb]{0.25,0.50,0.50}{##1}}}

\def\PYZbs{\char`\\}
\def\PYZus{\char`\_}
\def\PYZob{\char`\{}
\def\PYZcb{\char`\}}
\def\PYZca{\char`\^}
\def\PYZam{\char`\&}
\def\PYZlt{\char`\<}
\def\PYZgt{\char`\>}
\def\PYZsh{\char`\#}
\def\PYZpc{\char`\%}
\def\PYZdl{\char`\$}
\def\PYZhy{\char`\-}
\def\PYZsq{\char`\'}
\def\PYZdq{\char`\"}
\def\PYZti{\char`\~}
% for compatibility with earlier versions
\def\PYZat{@}
\def\PYZlb{[}
\def\PYZrb{]}
\makeatother


    % Exact colors from NB
    \definecolor{incolor}{rgb}{0.0, 0.0, 0.5}
    \definecolor{outcolor}{rgb}{0.545, 0.0, 0.0}



    
    % Prevent overflowing lines due to hard-to-break entities
    \sloppy 
    % Setup hyperref package
    \hypersetup{
      breaklinks=true,  % so long urls are correctly broken across lines
      colorlinks=true,
      urlcolor=urlcolor,
      linkcolor=linkcolor,
      citecolor=citecolor,
      }
    % Slightly bigger margins than the latex defaults
    
    \geometry{verbose,tmargin=1in,bmargin=1in,lmargin=1in,rmargin=1in}
    
    

    \begin{document}
    
    
    \maketitle
    
    

    
    \subsubsection{General rules:}\label{general-rules}

\begin{itemize}
\tightlist
\item
  For all figures that you generate, remember to add meaningful labels
  to the axes, and make a legend, if applicable.
\item
  Do not hard code constants, like number of samples, number of
  channels, etc in your program. These values should always be
  determined from the given data. This way, you can easily use the code
  to analyse other data sets.
\item
  Do not use high-level functions from toolboxes like scikit-learn.
\item
  Replace \emph{Template} by your \emph{FirstnameLastname} in the
  filename, or by \emph{Lastname1Lastname2} if you work in pairs.
\end{itemize}

    \section{BCI-IL - Exercise Sheet \#05}\label{bci-il---exercise-sheet-05}

    \paragraph{Name}\label{name}

    \begin{Verbatim}[commandchars=\\\{\}]
{\color{incolor}In [{\color{incolor}10}]:} \PY{o}{\PYZpc{}} \PY{n}{matplotlib} \PY{n}{inline}
         
         \PY{k+kn}{import} \PY{n+nn}{numpy} \PY{k}{as} \PY{n+nn}{np}
         \PY{k+kn}{import} \PY{n+nn}{scipy} \PY{k}{as} \PY{n+nn}{sp}
         \PY{k+kn}{import} \PY{n+nn}{scipy}\PY{n+nn}{.}\PY{n+nn}{signal}
         \PY{k+kn}{from} \PY{n+nn}{matplotlib} \PY{k}{import} \PY{n}{pyplot} \PY{k}{as} \PY{n}{plt}
         
         
         \PY{k+kn}{import} \PY{n+nn}{bci\PYZus{}minitoolbox} \PY{k}{as} \PY{n+nn}{bci}
\end{Verbatim}


    \subsection{Preparation: Load data}\label{preparation-load-data}

    \begin{Verbatim}[commandchars=\\\{\}]
{\color{incolor}In [{\color{incolor}11}]:} \PY{n}{fname}\PY{o}{=} \PY{l+s+s1}{\PYZsq{}}\PY{l+s+s1}{eyes\PYZus{}closed\PYZus{}VPal.npz}\PY{l+s+s1}{\PYZsq{}}
         \PY{n}{cnt}\PY{p}{,} \PY{n}{fs}\PY{p}{,} \PY{n}{clab}\PY{p}{,} \PY{n}{mnt} \PY{o}{=}\PY{n}{bci}\PY{o}{.}\PY{n}{load\PYZus{}data}\PY{p}{(}\PY{n}{fname}\PY{p}{)}
         \PY{n+nb}{print}\PY{p}{(}\PY{n}{cnt}\PY{o}{.}\PY{n}{shape}\PY{p}{)} \PY{c+c1}{\PYZsh{}118x5958}
         \PY{n+nb}{print}\PY{p}{(}\PY{n+nb}{len}\PY{p}{(}\PY{n}{clab}\PY{p}{)}\PY{p}{)} \PY{c+c1}{\PYZsh{}118}
         \PY{n+nb}{print}\PY{p}{(}\PY{n}{mnt}\PY{o}{.}\PY{n}{shape}\PY{p}{)} \PY{c+c1}{\PYZsh{}118x2}
\end{Verbatim}


    \begin{Verbatim}[commandchars=\\\{\}]
(118, 5958)
118
(118, 2)

    \end{Verbatim}

    \subsection{Exercise 1: PCA on raw data (3
points)}\label{exercise-1-pca-on-raw-data-3-points}

Make a scatter plot of the data with the two directions of largest
variance as coordinate axes. Then, depcit the projection vectors of
those two components as scalp maps (function \texttt{scalpmap} provided
in the \texttt{bbci\_minitoolbox}).

    \begin{Verbatim}[commandchars=\\\{\}]
{\color{incolor}In [{\color{incolor}26}]:} \PY{n}{X} \PY{o}{=} \PY{p}{(}\PY{n}{cnt} \PY{o}{\PYZhy{}} \PY{n}{np}\PY{o}{.}\PY{n}{mean}\PY{p}{(}\PY{n}{cnt}\PY{p}{)}\PY{p}{)}\PY{o}{/}\PY{n}{np}\PY{o}{.}\PY{n}{std}\PY{p}{(}\PY{n}{cnt}\PY{p}{)}       \PY{c+c1}{\PYZsh{} Standardization of variables as pre\PYZhy{}processing step}
         
         
         \PY{n}{C} \PY{o}{=} \PY{n}{np}\PY{o}{.}\PY{n}{cov}\PY{p}{(}\PY{n}{X}\PY{p}{)}
         \PY{n+nb}{print}\PY{p}{(}\PY{n}{cnt}\PY{o}{.}\PY{n}{shape}\PY{p}{)}
         \PY{n+nb}{print}\PY{p}{(}\PY{n}{C}\PY{o}{.}\PY{n}{shape}\PY{p}{)}
         \PY{n}{eig\PYZus{}vals}\PY{p}{,} \PY{n}{eig\PYZus{}vecs} \PY{o}{=}\PY{n}{np}\PY{o}{.}\PY{n}{linalg}\PY{o}{.}\PY{n}{eigh}\PY{p}{(}\PY{n}{C}\PY{p}{)}
         \PY{n+nb}{print} \PY{p}{(}\PY{n}{eig\PYZus{}vals}\PY{o}{.}\PY{n}{shape}\PY{p}{)}
         \PY{n+nb}{print}\PY{p}{(}\PY{n}{eig\PYZus{}vecs}\PY{o}{.}\PY{n}{shape}\PY{p}{)}
         
         \PY{c+c1}{\PYZsh{} Shae the two principal axes as a vector for}
         \PY{n}{matrix\PYZus{}w} \PY{o}{=} \PY{n}{np}\PY{o}{.}\PY{n}{hstack}\PY{p}{(}\PY{p}{(}\PY{n}{eig\PYZus{}vecs}\PY{p}{[}\PY{p}{:}\PY{p}{,}\PY{o}{\PYZhy{}}\PY{l+m+mi}{1}\PY{p}{]}\PY{o}{.}\PY{n}{reshape}\PY{p}{(}\PY{n+nb}{len}\PY{p}{(}\PY{n}{eig\PYZus{}vecs}\PY{p}{)}\PY{p}{,}\PY{l+m+mi}{1}\PY{p}{)}\PY{p}{,}\PY{n}{eig\PYZus{}vecs}\PY{p}{[}\PY{p}{:}\PY{p}{,}\PY{o}{\PYZhy{}}\PY{l+m+mi}{2}\PY{p}{]}\PY{o}{.}\PY{n}{reshape}\PY{p}{(}\PY{n+nb}{len}\PY{p}{(}\PY{n}{eig\PYZus{}vecs}\PY{p}{)}\PY{p}{,}\PY{l+m+mi}{1}\PY{p}{)}\PY{p}{)}\PY{p}{)}
         
         \PY{n+nb}{print} \PY{p}{(} \PY{n}{np}\PY{o}{.}\PY{n}{shape}\PY{p}{(}\PY{n}{matrix\PYZus{}w}\PY{p}{)}\PY{p}{)}
         
         \PY{c+c1}{\PYZsh{}Transform the data to the principal axes co\PYZhy{}ordinate system}
         \PY{n}{Y} \PY{o}{=} \PY{n}{X}\PY{o}{.}\PY{n}{T}\PY{o}{.}\PY{n}{dot}\PY{p}{(}\PY{n}{matrix\PYZus{}w}\PY{p}{)}
         
         \PY{n+nb}{print} \PY{p}{(} \PY{n}{np}\PY{o}{.}\PY{n}{shape} \PY{p}{(}\PY{n}{Y}\PY{p}{)}\PY{p}{)}
         
         \PY{n}{plt}\PY{o}{.}\PY{n}{figure}\PY{p}{(}\PY{n}{figsize} \PY{o}{=} \PY{p}{(}\PY{l+m+mi}{12}\PY{p}{,}\PY{l+m+mi}{12}\PY{p}{)}\PY{p}{)}
         \PY{c+c1}{\PYZsh{} plt.scatter(eig\PYZus{}vecs[:,\PYZhy{}1], eig\PYZus{}vecs[:,\PYZhy{}2])}
         \PY{n}{plt}\PY{o}{.}\PY{n}{scatter}\PY{p}{(}\PY{n}{Y}\PY{p}{[}\PY{p}{:}\PY{p}{,}\PY{l+m+mi}{0}\PY{p}{]}\PY{p}{,} \PY{n}{Y}\PY{p}{[}\PY{p}{:}\PY{p}{,}\PY{l+m+mi}{1}\PY{p}{]}\PY{p}{)}
         \PY{n}{plt}\PY{o}{.}\PY{n}{xlabel}\PY{p}{(}\PY{l+s+s2}{\PYZdq{}}\PY{l+s+s2}{First largest variance}\PY{l+s+s2}{\PYZdq{}}\PY{p}{)}
         \PY{n}{plt}\PY{o}{.}\PY{n}{ylabel}\PY{p}{(}\PY{l+s+s2}{\PYZdq{}}\PY{l+s+s2}{Second largest variance}\PY{l+s+s2}{\PYZdq{}}\PY{p}{)}
         
         \PY{n}{plt}\PY{o}{.}\PY{n}{figure}\PY{p}{(}\PY{n}{figsize} \PY{o}{=} \PY{p}{(}\PY{l+m+mi}{12}\PY{p}{,}\PY{l+m+mi}{6}\PY{p}{)}\PY{p}{)}
         \PY{n}{plt}\PY{o}{.}\PY{n}{subplot}\PY{p}{(}\PY{l+m+mi}{121}\PY{p}{)}
         \PY{n}{bci}\PY{o}{.}\PY{n}{scalpmap}\PY{p}{(}\PY{n}{mnt}\PY{p}{,}\PY{n}{eig\PYZus{}vecs}\PY{p}{[}\PY{p}{:}\PY{p}{,}\PY{o}{\PYZhy{}}\PY{l+m+mi}{1}\PY{p}{]}\PY{p}{)}
         \PY{n}{plt}\PY{o}{.}\PY{n}{subplot}\PY{p}{(}\PY{l+m+mi}{122}\PY{p}{)}
         \PY{n}{bci}\PY{o}{.}\PY{n}{scalpmap}\PY{p}{(}\PY{n}{mnt}\PY{p}{,}\PY{n}{eig\PYZus{}vecs}\PY{p}{[}\PY{p}{:}\PY{p}{,}\PY{o}{\PYZhy{}}\PY{l+m+mi}{2}\PY{p}{]}\PY{p}{)}
\end{Verbatim}


    \begin{Verbatim}[commandchars=\\\{\}]
(118, 5958)
(118, 118)
(118,)
(118, 118)
(118, 2)
(5958, 2)

    \end{Verbatim}

    \begin{center}
    \adjustimage{max size={0.9\linewidth}{0.9\paperheight}}{output_7_1.png}
    \end{center}
    { \hspace*{\fill} \\}
    
    \begin{center}
    \adjustimage{max size={0.9\linewidth}{0.9\paperheight}}{output_7_2.png}
    \end{center}
    { \hspace*{\fill} \\}
    
    \subsection{Exercise 2: Artifact to signal ratio with PCA (5
points)}\label{exercise-2-artifact-to-signal-ratio-with-pca-5-points}

For this task we assume that the two components from Ex. \#01 reflect
eye movements, while all other components do not contain artifacts from
eye movement. If you did not succeed with Ex. \#01, chose an arbitrary
component.

Determine for each channel which proportion of the overall variance is
caused by eye movements and plot this information as a scalp map. Also,
calculate the Signal-To-Noise ratio (SNR) per channel in Decibel (dB).

    \begin{Verbatim}[commandchars=\\\{\}]
{\color{incolor}In [{\color{incolor}4}]:} \PY{n}{idx}\PY{o}{=}\PY{p}{[}\PY{o}{\PYZhy{}}\PY{l+m+mi}{1}\PY{p}{,}\PY{o}{\PYZhy{}}\PY{l+m+mi}{2}\PY{p}{]}
        \PY{n}{cnt\PYZus{}s} \PY{o}{=} \PY{n}{V}\PY{p}{[}\PY{p}{:}\PY{p}{,} \PY{n}{idx}\PY{p}{]}\PY{o}{.}\PY{n}{T}\PY{n+nd}{@cnt} \PY{c+c1}{\PYZsh{} W.TX}
        \PY{n+nb}{print}\PY{p}{(}\PY{n}{cnt\PYZus{}s}\PY{o}{.}\PY{n}{shape}\PY{p}{)}
        
        \PY{n}{x\PYZus{}eyes}\PY{o}{=}\PY{n}{V}\PY{p}{[}\PY{p}{:}\PY{p}{,}\PY{n}{idx}\PY{p}{]}\PY{n+nd}{@cnt\PYZus{}s} \PY{c+c1}{\PYZsh{}AS}
        \PY{n+nb}{print}\PY{p}{(}\PY{n}{x\PYZus{}eyes}\PY{o}{.}\PY{n}{shape}\PY{p}{)}
        
        \PY{n}{cnt\PYZus{}artifree}\PY{o}{=}\PY{n}{cnt}\PY{o}{\PYZhy{}}\PY{n}{x\PYZus{}eyes}
        \PY{n}{C\PYZus{}clean} \PY{o}{=} \PY{n}{np}\PY{o}{.}\PY{n}{cov}\PY{p}{(}\PY{n}{cnt\PYZus{}artifree}\PY{p}{)}
        \PY{n}{C\PYZus{}noise} \PY{o}{=} \PY{n}{np}\PY{o}{.}\PY{n}{cov}\PY{p}{(}\PY{n}{x\PYZus{}eyes}\PY{p}{)}
        
        \PY{n}{a} \PY{o}{=} \PY{n}{np}\PY{o}{.}\PY{n}{diagonal}\PY{p}{(}\PY{n}{C}\PY{p}{)}\PY{o}{.}\PY{n}{copy}\PY{p}{(}\PY{p}{)}
        \PY{n}{b} \PY{o}{=} \PY{n}{np}\PY{o}{.}\PY{n}{diagonal}\PY{p}{(}\PY{n}{C\PYZus{}clean}\PY{p}{)}\PY{o}{.}\PY{n}{copy}\PY{p}{(}\PY{p}{)}
        \PY{n}{c} \PY{o}{=} \PY{n}{np}\PY{o}{.}\PY{n}{diagonal}\PY{p}{(}\PY{n}{C\PYZus{}noise}\PY{p}{)}\PY{o}{.}\PY{n}{copy}\PY{p}{(}\PY{p}{)}
        
        \PY{n}{proportion} \PY{o}{=} \PY{n}{np}\PY{o}{.}\PY{n}{divide}\PY{p}{(}\PY{n}{b}\PY{p}{,}\PY{n}{a}\PY{p}{)}
        \PY{n}{SNR} \PY{o}{=} \PY{n}{np}\PY{o}{.}\PY{n}{divide}\PY{p}{(}\PY{n}{b}\PY{p}{,}\PY{n}{c}\PY{p}{)}
        \PY{n+nb}{print}\PY{p}{(}\PY{n}{SNR}\PY{p}{)}
        \PY{n}{SNR\PYZus{}db} \PY{o}{=} \PY{l+m+mi}{10} \PY{o}{*} \PY{n}{np}\PY{o}{.}\PY{n}{log10}\PY{p}{(}\PY{n}{SNR}\PY{p}{)}
        \PY{n+nb}{print}\PY{p}{(}\PY{n}{SNR\PYZus{}db}\PY{p}{)}
        
        \PY{n}{plt}\PY{o}{.}\PY{n}{figure}\PY{p}{(}\PY{p}{)}
        \PY{n}{bci}\PY{o}{.}\PY{n}{scalpmap}\PY{p}{(}\PY{n}{mnt}\PY{p}{,} \PY{n}{proportion}\PY{o}{*}\PY{l+m+mi}{100}\PY{p}{,} \PY{n}{cb\PYZus{}label} \PY{o}{=} \PY{l+s+s1}{\PYZsq{}}\PY{l+s+s1}{\PYZpc{}}\PY{l+s+s1}{\PYZsq{}}\PY{p}{)}
        \PY{n}{plt}\PY{o}{.}\PY{n}{title}\PY{p}{(}\PY{l+s+s1}{\PYZsq{}}\PY{l+s+s1}{Percent of the overall variance per channel caused by eye movements}\PY{l+s+s1}{\PYZsq{}}\PY{p}{)}
        
        \PY{n}{plt}\PY{o}{.}\PY{n}{figure}\PY{p}{(}\PY{p}{)}
        \PY{n}{bci}\PY{o}{.}\PY{n}{scalpmap}\PY{p}{(}\PY{n}{mnt}\PY{p}{,}\PY{n}{SNR\PYZus{}db}\PY{p}{,}\PY{n}{cb\PYZus{}label} \PY{o}{=} \PY{l+s+s1}{\PYZsq{}}\PY{l+s+s1}{dB}\PY{l+s+s1}{\PYZsq{}}\PY{p}{)}
        \PY{n}{plt}\PY{o}{.}\PY{n}{title}\PY{p}{(}\PY{l+s+s1}{\PYZsq{}}\PY{l+s+s1}{Signal\PYZhy{}To\PYZhy{}Noise Ratio per channel}\PY{l+s+s1}{\PYZsq{}}\PY{p}{)}
\end{Verbatim}


    \begin{Verbatim}[commandchars=\\\{\}]
(2, 5958)
(118, 5958)
[0.04117148 0.0125623  0.01103606 0.02875167 0.06670432 0.04908415
 0.02053914 0.01766199 0.10575103 0.02321465 0.03645448 0.0551194
 0.02710009 0.10572214 0.02903875 0.01586516 0.03044617 0.03476235
 0.03316477 0.0345375  0.02536871 0.1546038  0.11457591 0.03407083
 0.01669189 0.03008247 0.03476775 0.03476789 0.03257575 0.06790173
 0.1815352  0.33189666 0.03324921 0.02544456 0.03201951 0.042952
 0.03113454 0.05727149 0.06628821 0.34167808 0.18550278 0.15053264
 0.02333207 0.0350121  0.07335829 0.0384789  0.08883425 0.06889233
 0.56887516 0.10680973 0.02950028 0.02852497 0.04389717 0.04682786
 0.04598268 0.04243431 0.07489315 0.26481501 0.09139603 0.02883172
 0.03315253 0.04121504 0.06066977 0.03720396 0.06680252 0.07801678
 0.22530586 0.10383847 0.04050055 0.04997187 0.05346817 0.03897511
 0.04586196 0.03832836 0.05847714 0.09586087 0.10841644 0.09057888
 0.04700158 0.06498477 0.04470261 0.07471549 0.03813472 0.04294965
 0.12570646 0.1779728  0.18213524 0.13553075 0.0691771  0.06446931
 0.04922145 0.0608588  0.07228524 0.07409783 0.16023814 0.17666462
 0.20288876 0.10206659 0.06081616 0.13406612 0.11794077 0.19371778
 0.20680319 0.18393664 0.05889886 0.07807396 0.11863616 0.1937559
 0.25848674 0.06202598 0.08589326 0.11991543 0.06443613 0.19104273
 0.08932338 0.1246812  0.06744156 0.11992543]
[-13.85403486 -19.00930703 -19.57185846 -15.41336999 -11.75846036
 -13.09058683 -16.8741766  -17.52960244  -9.75715408 -16.34237904
 -14.38249058 -12.58695514 -15.67029274  -9.75834061 -15.37022053
 -17.99555586 -15.16467342 -14.58890885 -14.79323024 -14.61709073
 -15.95701614  -8.10779841  -9.40906683 -14.67617279 -17.77494471
 -15.21686527 -14.58823384 -14.58821649 -14.87105542 -11.68119153
  -7.41039149  -4.78997112 -14.78218616 -15.94405102 -14.94585325
 -13.67016636 -15.06757597 -12.42061524 -11.78563722  -4.66382882
  -7.31649576  -8.22369318 -16.32046658 -14.55781819 -11.34550818
 -14.14777303 -10.5141955  -11.61829149  -2.4498303   -9.71389194
 -15.30173924 -15.4477476  -13.57563485 -13.29495719 -13.37405709
 -13.72282829 -11.25557902  -5.77057397 -10.39072662 -15.40129467
 -14.79483268 -13.84944311 -12.17027635 -14.29410882 -11.75207171
 -11.07811974  -6.47227514  -9.83641716 -13.9253909  -13.01274437
 -12.71904716 -14.09212658 -13.38547343 -14.16479797 -12.3301391
 -10.18358639  -9.64904856 -10.42973052 -13.27887555 -11.87188392
 -13.49667117 -11.26589354 -14.18679398 -13.67040361  -9.0064242
  -7.49646373  -7.39606025  -8.6796216  -11.60037625 -11.90646983
 -13.07845565 -12.15676594 -11.40950394 -11.30194528  -7.95234096
  -7.52850422  -6.92742018  -9.91116384 -12.1598102   -8.72680967
  -9.28336028  -7.12830528  -6.8444277   -7.35331756 -12.29893119
 -11.07493819  -9.25782924  -7.12745062  -5.8756173  -12.07426371
 -10.66040894  -9.21124929 -11.90870557  -7.18869489 -10.4903487
  -9.04199012 -11.71072376  -9.21088721]

    \end{Verbatim}

\begin{Verbatim}[commandchars=\\\{\}]
{\color{outcolor}Out[{\color{outcolor}4}]:} Text(0.5,1,'Signal-To-Noise Ratio per channel')
\end{Verbatim}
            
    \begin{center}
    \adjustimage{max size={0.9\linewidth}{0.9\paperheight}}{output_9_2.png}
    \end{center}
    { \hspace*{\fill} \\}
    
    \begin{center}
    \adjustimage{max size={0.9\linewidth}{0.9\paperheight}}{output_9_3.png}
    \end{center}
    { \hspace*{\fill} \\}
    
    \subsection{Preparation: Load data}\label{preparation-load-data}

    \begin{Verbatim}[commandchars=\\\{\}]
{\color{incolor}In [{\color{incolor}37}]:} \PY{n}{fname} \PY{o}{=} \PY{l+s+s1}{\PYZsq{}}\PY{l+s+s1}{erp\PYZus{}hexVPsag.npz}\PY{l+s+s1}{\PYZsq{}}
         \PY{n}{cnt}\PY{p}{,} \PY{n}{fs}\PY{p}{,} \PY{n}{clab}\PY{p}{,} \PY{n}{mnt}\PY{p}{,} \PY{n}{mrk\PYZus{}pos}\PY{p}{,} \PY{n}{mrk\PYZus{}class}\PY{p}{,} \PY{n}{mrk\PYZus{}className} \PY{o}{=} \PY{n}{bci}\PY{o}{.}\PY{n}{load\PYZus{}data}\PY{p}{(}\PY{n}{fname}\PY{p}{)}
\end{Verbatim}


    \begin{Verbatim}[commandchars=\\\{\}]
(55, 2)

    \end{Verbatim}

    \subsection{Exercise 3: Artificial EEG data (7
points)}\label{exercise-3-artificial-eeg-data-7-points}

Generate one trial of artificial, stereotypical EEG data (1000 ms, 55
channels) out of the data set of sheet \#01. The trial should contain a
'clean' target ERP composed of an N2 component (the one negatively
peaking at 310 ms in the data on sheet \#01) and a P3 component (the one
peaking at 380 ms in the data on sheet \#01). Both components should
have their typical spatial distribution. To this extent, extract the
corresponding scalp patterns at the peaks of the average ERPs, calculate
the filters, use them to isolate the components from the average ERP and
then project them back into the EEG space. Plot the artificial EEG (the
backprojected ERP) in channels PO7 and Cz and the scalp patterns
correpsonding to the N2 and P3.

    \begin{Verbatim}[commandchars=\\\{\}]
{\color{incolor}In [{\color{incolor}126}]:} \PY{n}{ival}\PY{o}{=} \PY{p}{[}\PY{o}{\PYZhy{}}\PY{l+m+mi}{100}\PY{p}{,} \PY{l+m+mi}{1000}\PY{p}{]}
          \PY{n}{ref\PYZus{}ival}\PY{o}{=} \PY{p}{[}\PY{o}{\PYZhy{}}\PY{l+m+mi}{100}\PY{p}{,} \PY{l+m+mi}{0}\PY{p}{]}
          
          \PY{c+c1}{\PYZsh{} Segment continuous data into epochs:}
          \PY{n}{epo}\PY{p}{,} \PY{n}{epo\PYZus{}t} \PY{o}{=} \PY{n}{bci}\PY{o}{.}\PY{n}{makeepochs}\PY{p}{(}\PY{n}{cnt}\PY{p}{,} \PY{n}{fs}\PY{p}{,} \PY{n}{mrk\PYZus{}pos}\PY{p}{,} \PY{n}{ival}\PY{p}{)}
          \PY{c+c1}{\PYZsh{} Baseline correction:}
          \PY{n}{epo} \PY{o}{=} \PY{n}{bci}\PY{o}{.}\PY{n}{baseline}\PY{p}{(}\PY{n}{epo}\PY{p}{,} \PY{n}{epo\PYZus{}t}\PY{p}{,} \PY{n}{ref\PYZus{}ival}\PY{p}{)}
          
          \PY{n}{erp} \PY{o}{=} \PY{n}{np}\PY{o}{.}\PY{n}{mean}\PY{p}{(}\PY{n}{epo}\PY{p}{[}\PY{p}{:}\PY{p}{,} \PY{p}{:}\PY{p}{,} \PY{n}{mrk\PYZus{}class}\PY{o}{==}\PY{l+m+mi}{0}\PY{p}{]}\PY{p}{,} \PY{n}{axis}\PY{o}{=}\PY{l+m+mi}{2}\PY{p}{)}
          
          \PY{n}{erp\PYZus{}1st\PYZus{}plot} \PY{o}{=} \PY{n}{erp}\PY{p}{[}\PY{p}{:}\PY{p}{,}\PY{n}{clab}\PY{o}{.}\PY{n}{index}\PY{p}{(}\PY{l+s+s1}{\PYZsq{}}\PY{l+s+s1}{Cz}\PY{l+s+s1}{\PYZsq{}}\PY{p}{)}\PY{p}{]} \PY{o}{+} \PY{n}{erp}\PY{p}{[}\PY{p}{:}\PY{p}{,}\PY{n}{clab}\PY{o}{.}\PY{n}{index}\PY{p}{(}\PY{l+s+s1}{\PYZsq{}}\PY{l+s+s1}{PO7}\PY{l+s+s1}{\PYZsq{}}\PY{p}{)}\PY{p}{]}
          \PY{n}{maxx} \PY{o}{=} \PY{n}{np}\PY{o}{.}\PY{n}{where}\PY{p}{(}\PY{n}{erp\PYZus{}1st\PYZus{}plot} \PY{o}{==} \PY{n}{erp\PYZus{}1st\PYZus{}plot}\PY{o}{.}\PY{n}{max}\PY{p}{(}\PY{p}{)}\PY{p}{)}\PY{p}{[}\PY{l+m+mi}{0}\PY{p}{]}\PY{p}{[}\PY{l+m+mi}{0}\PY{p}{]}
          \PY{n}{minn} \PY{o}{=} \PY{n}{np}\PY{o}{.}\PY{n}{where}\PY{p}{(}\PY{n}{erp\PYZus{}1st\PYZus{}plot} \PY{o}{==} \PY{n}{erp\PYZus{}1st\PYZus{}plot}\PY{o}{.}\PY{n}{min}\PY{p}{(}\PY{p}{)}\PY{p}{)}\PY{p}{[}\PY{l+m+mi}{0}\PY{p}{]}\PY{p}{[}\PY{l+m+mi}{0}\PY{p}{]}
          
          \PY{n}{plt}\PY{o}{.}\PY{n}{figure}\PY{p}{(}\PY{p}{)}
          \PY{n}{plt}\PY{o}{.}\PY{n}{plot}\PY{p}{(}\PY{n}{epo\PYZus{}t}\PY{p}{,} \PY{n}{erp\PYZus{}1st\PYZus{}plot}\PY{p}{)}
          \PY{n}{plt}\PY{o}{.}\PY{n}{title}\PY{p}{(}\PY{l+s+s1}{\PYZsq{}}\PY{l+s+s1}{Target ERP composed of N2 \PYZam{} P3 components at 370 \PYZam{} 380ms}\PY{l+s+s1}{\PYZsq{}}\PY{p}{)}
          \PY{n}{plt}\PY{o}{.}\PY{n}{xlabel}\PY{p}{(}\PY{l+s+s1}{\PYZsq{}}\PY{l+s+s1}{time  [ms]}\PY{l+s+s1}{\PYZsq{}}\PY{p}{)}
          \PY{n}{plt}\PY{o}{.}\PY{n}{ylabel}\PY{p}{(}\PY{l+s+s1}{\PYZsq{}}\PY{l+s+s1}{potential  [uV]}\PY{l+s+s1}{\PYZsq{}}\PY{p}{)}
          
          \PY{n}{plt}\PY{o}{.}\PY{n}{figure}\PY{p}{(}\PY{p}{)}
          \PY{n}{bci}\PY{o}{.}\PY{n}{scalpmap}\PY{p}{(}\PY{n}{mnt}\PY{p}{,}\PY{n}{erp}\PY{p}{[}\PY{n}{maxx}\PY{p}{,}\PY{p}{:}\PY{p}{]} \PY{p}{,} \PY{n}{cb\PYZus{}label} \PY{o}{=} \PY{l+s+s1}{\PYZsq{}}\PY{l+s+s1}{potential  [uV]}\PY{l+s+s1}{\PYZsq{}}\PY{p}{)}
          \PY{n}{plt}\PY{o}{.}\PY{n}{title}\PY{p}{(}\PY{l+s+s1}{\PYZsq{}}\PY{l+s+s1}{Scalp pattern of P3}\PY{l+s+s1}{\PYZsq{}}\PY{p}{)}
          
          \PY{n}{plt}\PY{o}{.}\PY{n}{figure}\PY{p}{(}\PY{p}{)}
          \PY{n}{bci}\PY{o}{.}\PY{n}{scalpmap}\PY{p}{(}\PY{n}{mnt}\PY{p}{,}\PY{n}{erp}\PY{p}{[}\PY{n}{minn}\PY{p}{,}\PY{p}{:}\PY{p}{]} \PY{p}{,} \PY{n}{cb\PYZus{}label} \PY{o}{=} \PY{l+s+s1}{\PYZsq{}}\PY{l+s+s1}{potential  [uV]}\PY{l+s+s1}{\PYZsq{}}\PY{p}{)}
          \PY{n}{plt}\PY{o}{.}\PY{n}{title}\PY{p}{(}\PY{l+s+s1}{\PYZsq{}}\PY{l+s+s1}{Scalp pattern of N2}\PY{l+s+s1}{\PYZsq{}}\PY{p}{)}
          
          
          \PY{c+c1}{\PYZsh{}\PYZsh{} calculate the filters, use them to isolate the components from the average ERP and then project them back into the EEG space.}
          
          \PY{c+c1}{\PYZsh{} no idea what to do here at all, didnt quite get the whole concept of filters and there is no available information}
          \PY{c+c1}{\PYZsh{} about that from the slides, so no idea how to proceed further}
\end{Verbatim}


\begin{Verbatim}[commandchars=\\\{\}]
{\color{outcolor}Out[{\color{outcolor}126}]:} Text(0.5,1,'Scalp pattern of N2')
\end{Verbatim}
            
    \begin{center}
    \adjustimage{max size={0.9\linewidth}{0.9\paperheight}}{output_13_1.png}
    \end{center}
    { \hspace*{\fill} \\}
    
    \begin{center}
    \adjustimage{max size={0.9\linewidth}{0.9\paperheight}}{output_13_2.png}
    \end{center}
    { \hspace*{\fill} \\}
    
    \begin{center}
    \adjustimage{max size={0.9\linewidth}{0.9\paperheight}}{output_13_3.png}
    \end{center}
    { \hspace*{\fill} \\}
    
    \begin{Verbatim}[commandchars=\\\{\}]
{\color{incolor}In [{\color{incolor} }]:} \PY{c+c1}{\PYZsh{}\PYZsh{}}
\end{Verbatim}


    \begin{Verbatim}[commandchars=\\\{\}]
{\color{incolor}In [{\color{incolor} }]:} \PY{c+c1}{\PYZsh{}\PYZsh{}}
\end{Verbatim}



    % Add a bibliography block to the postdoc
    
    
    
    \end{document}
